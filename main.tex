%%%%%%%%%% !TEX program = pdflatex %%%%%%%%%%
% !TEX program = pdflatex
% !Mode:: "TeX:UTF-8"
% \def\usewhat{pdflatex}  % 定义编译方式 dvipdfmx 或者 pdflatex

% """
% Created on 03/03/2023
% anoi.
% @author: Kang Xiatao (kangxiatao@gmail.com)
% """

%\setlength{\baselineskip}{20pt}
%\setlength{\headheight}{25pt}
\documentclass[a4paper,12pt,openany,twoside]{book}

                                           % 如果论文超过60页 可以使用twoside 双面打印
\input{setup/package}                      % 定义本文所使用宏包
\graphicspath{{figures/}}                  % 定义所有的.eps文件在figures子目录下
\begin{document}                           % 开始全文
\begin{CJK*}{UTF8}{song}                   % 开始中文字体使用
% !Mode:: "TeX:UTF-8"
%  Authors: 杜家宜   Jiayi Du: Max_dujiayi@gmail.com     湖南大学2010级计算机科学与技术专业博士生

% """
% Edited on 03/10/2023
% anoi.
% @author: Kang Xiatao (kangxiatao@gmail.com)
% """

%%%%%%%%%% Fonts Definition and Basics %%%%%%%%%%%%%%%%%
\newcommand{\song}{\CJKfamily{song}}    % 宋体
\newcommand{\fs}{\CJKfamily{fs}}        % 仿宋体
\newcommand{\kai}{\CJKfamily{kai}}      % 楷体
\newcommand{\hei}{\CJKfamily{hei}}      % 黑体
\newcommand{\li}{\CJKfamily{li}}        % 隶书
\newcommand{\yihao}{\fontsize{26pt}{26pt}\selectfont}       % 一号, 1.倍行距
\newcommand{\xiaoyi}{\fontsize{24pt}{24pt}\selectfont}      % 小一, 1.倍行距
\newcommand{\erhao}{\fontsize{22pt}{22pt}\selectfont}       % 二号, 1.倍行距
\newcommand{\xiaoer}{\fontsize{18pt}{18pt}\selectfont}      % 小二, 单倍行距
\newcommand{\sanhao}{\fontsize{16pt}{16pt}\selectfont}      % 三号, 1.倍行距
\newcommand{\xiaosan}{\fontsize{15pt}{15pt}\selectfont}     % 小三, 1.倍行距
\newcommand{\sihao}{\fontsize{14pt}{14pt}\selectfont}       % 四号, 1.0倍行距
\newcommand{\xiaosi}{\fontsize{12pt}{12pt}\selectfont}      % 小四, 1.倍行距
\newcommand{\wuhao}{\fontsize{10.5pt}{10.5pt}\selectfont}   % 五号, 单倍行距
\newcommand{\xiaowu}{\fontsize{9pt}{9pt}\selectfont}        % 小五, 单倍行距
\newcommand{\xiaoxiao}{\fontsize{4pt}{4pt}\selectfont}        % 小小  用于调节行距

% 设置一个字,对应宽度压缩至 75%
% \newcommand{\bianhei}{\setCJKmainfont{hei}[FakeStretch=0.75]}    

% \setlength{\headheight}{15pt}
%\CJKcaption{gb_452}
\CJKtilde  % 重新定义了波浪符~的意义
\newcommand\prechaptername{第}
\newcommand\postchaptername{章}

% 调整罗列环境的布局 % Modified by Li Jianmin
\setitemize{leftmargin=0em,itemindent=3em,itemsep=0em,partopsep=0em,parsep=0em,topsep=-0em}
\setenumerate{leftmargin=0em,itemindent=3em,itemsep=0em,partopsep=0em,parsep=0em,topsep=0em}


%避免宏包 hyperref 和 arydshln 不兼容带来的目录链接失效的问题。
\def\temp{\relax}
\let\temp\addcontentsline
\gdef\addcontentsline{\phantomsection\temp}

% 自定义项目列表标签及格式 \begin{publist} 列表项 \end{publist}
\newcounter{pubctr} % 自定义新计数器
\newenvironment{publist}{%%%%% 定义新环境
\begin{list}{[\arabic{pubctr}]} %% 标签格式
    {
     \usecounter{pubctr}
     \setlength{\leftmargin}{2em}     % 左边界 \leftmargin =\itemindent + \labelwidth + \labelsep
     \setlength{\itemindent}{0em}     % 标号缩进量
     \setlength{\labelsep}{1em}       % 标号和列表项之间的距离,默认0.5em
     \setlength{\rightmargin}{0em}    % 右边界
     \setlength{\topsep}{0ex}         % 列表到上下文的垂直距离
     \setlength{\parsep}{0ex}         % 段落间距
     \setlength{\itemsep}{0ex}        % 标签间距
     \setlength{\listparindent}{0pt} % 段落缩进量
    }}
{\end{list}}%%%%%


\makeatletter
\renewcommand\normalsize{
  \@setfontsize\normalsize{12pt}{12pt} % 小四对应12pt
  \setlength\abovedisplayskip{4pt}
  \setlength\abovedisplayshortskip{4pt}
  \setlength\belowdisplayskip{\abovedisplayskip}
  \setlength\belowdisplayshortskip{\abovedisplayshortskip}
    \let\@listi\@listI}
% 行间距  1.5 * 1.3
\def\defaultfont{\renewcommand{\baselinestretch}{1.95}\normalsize\selectfont}
\def\sk20font{\renewcommand{\baselinestretch}{1.6}\normalsize\selectfont}


% 设置行距和段落间垂直距离

% \setlength{\baselineskip}{20pt}
\renewcommand{\CJKglue}{\hskip 0.5pt plus \baselineskip} %加大字间距,使每行35个字


\makeatother

%%%%%%%%%%%%% Contents %%%%%%%%%%%%%%%%%
\renewcommand{\contentsname}{\xiaosan\hei 目\quad录}
\setcounter{tocdepth}{2}
\titlecontents{chapter}[0em]{\xiaosi\hei}%
             {\prechaptername~~\thecontentslabel~~\postchaptername~~~}{} %
             {\titlerule*[5pt]{$\cdot$}\xiaosi\contentspage}
\titlecontents{section}[2.5em]{\xiaosi\song} %
            {\thecontentslabel\quad}{} %
            {\hspace{.25em}\titlerule*[5pt]{$\cdot$}\xiaosi\contentspage}
\titlecontents{subsection}[3.25em]{\xiaosi\song} %
            {\thecontentslabel\quad}{} %
            {\hspace{.25em}\titlerule*[5pt]{$\cdot$}\xiaosi\contentspage}
\renewcommand{\cftdotsep}{1.1}
\renewcommand{\listfigurename}{\xiaosan\hei 插图索引}
\setcounter{lofdepth}{1}
%\titlefigures{chapter}[1em]{\xiaosi\hei}%
%             {\prechaptername~~\thecontentslabel~~\postchaptername~~~}{} %
%            {\titlerule*[5pt]{$\cdot$}\xiaosi\contentspage}
\renewcommand{\listtablename}{\xiaosan\hei 附表索引}


%%删除表格和插图因章不同中的空行%%%
\makeatletter
\def\@chapter[#1]#2{\ifnum \c@secnumdepth >\m@ne
                       \if@mainmatter
                         \refstepcounter{chapter}%
                         \typeout{\@chapapp\space\thechapter.}%
                         \addcontentsline{toc}{chapter}%
                                   {\protect\numberline{\thechapter}#1}%
                       \else
                         \addcontentsline{toc}{chapter}{#1}%
                       \fi
                    \else
                      \addcontentsline{toc}{chapter}{#1}%
                    \fi
                    \chaptermark{#1}%
                    \if@twocolumn
                      \@topnewpage[\@makechapterhead{#2}]%
                    \else
                      \@makechapterhead{#2}%
                      \@afterheading
                    \fi}
\makeatother


%%%%%%%%%% Chapter and Section %%%%%%%%%%%%%%%%%
\setcounter{secnumdepth}{4}
\setlength{\parindent}{2em}
\renewcommand{\chaptername}{\prechaptername\arabic{chapter}\postchaptername}
% \renewcommand{\chaptername}{\prechaptername\chinese{chapter}\postchaptername}
\titleformat{\chapter}{\centering\sihao\hei}{\chaptername}{1em}{}
\titlespacing{\chapter}{0pt}{14pt}{14pt}
\titleformat{\section}{\xiaosi\hei}{\thesection}{1em}{}
\titlespacing{\section}{0pt}{12pt}{12pt}
\titleformat{\subsection}{\xiaosi\hei}{\thesubsection}{0.5em}{}
\titlespacing{\subsection}{0pt}{6pt}{6pt}
\titleformat{\subsubsection}{\xiaosi\hei}{\thesubsubsection}{0.5em}{}
\titlespacing{\subsubsection}{0pt}{6pt}{6pt}

%%%%%%%%%% Table, Figure and Equation %%%%%%%%%%%%%%%%%
\renewcommand{\tablename}{表} % 插表题头
\renewcommand{\figurename}{图} % 插图题头
\renewcommand{\thefigure}{\arabic{chapter}.\arabic{figure}} % 使图编号为 7.1 的格式 %\protect{~}
\renewcommand{\thetable}{\arabic{chapter}.\arabic{table}}%使表编号为 7.1 的格式
\renewcommand{\theequation}{\arabic{chapter}.\arabic{equation}}%使公式编号为 7.1 的格式
\renewcommand{\thesubfigure}{\alph{subfigure})}%使子图编号为a)的格式
\renewcommand{\thesubtable}{\alph{subtable})} %使子表编号为a)的格式
\makeatletter
\renewcommand{\p@subfigure}{\thefigure~} %使子图引用为 7.1 a) 的格式,母图编号和子图编号之间用~ 加一个空格
\makeatother


%% 定制浮动图形和表格标题样式
\makeatletter
\long\def\@makecaption#1#2{%
  %  \vskip\abovecaptionskip
   \vskip 6pt
   \sbox\@tempboxa{\centering\wuhao\song{#1~~#2} }%
   \ifdim \wd\@tempboxa >\hsize
     \centering\wuhao\song{#1~~#2} \par
   \else
     \global \@minipagefalse
     \hb@xt@\hsize{\hfil\box\@tempboxa\hfil}%
   \fi
  %  \vskip\belowcaptionskip}
   \vskip 6pt}
\makeatother
\captiondelim{~~~~} %用来控制longtable表头分隔符

%%%%%%%%%% Theorem Environment %%%%%%%%%%%%%%%%%
\theoremstyle{plain}
\theorembodyfont{\xiaosi\song}%\rmfamily}
\theoremheaderfont{\xiaosi\hei}%\rmfamily}
\setlength{\theorempreskipamount}{0em} %调整定理环境与上文的距离
\setlength{\theorempostskipamount}{0em} %调整定理环境与下文的距离
\newtheorem{theorem}{定理~}[chapter]
\newtheorem{lemma}{引理~}[chapter]
\newtheorem{axiom}{公理~}[chapter]
\newtheorem{proposition}{命题~}[chapter]
\newtheorem{corollary}{推论~}[chapter]
\newtheorem{definition}{\hskip 2em 定义~}[chapter]
\newtheorem{conjecture}{猜想~}[chapter]
\newtheorem{example}{例~}[chapter]
\newtheorem{remark}{注~}[chapter]
\floatname{algorithm}{算法}% 将英文的algorithm改为算法
\renewcommand{\algorithmicrequire}{\textbf{Input:}}
\renewcommand{\algorithmicensure}{\textbf{Output:}}
\newcommand{\tabincell}[2]{\begin{tabular}{@{}#1@{}}#2\end{tabular}}%表格合并
\newenvironment{proof}{\noindent{\hei 证明:}}{\hfill $ \square $ \vskip 4mm}
\theoremsymbol{$\square$}

%%%%%%%%%% Page: number, header and footer  页码%%%%%%%%%%%%%%%%%

%\frontmatter 或 \pagenumbering{roman}
%\mainmatter 或 \pagenumbering{arabic}
\makeatletter
\renewcommand\frontmatter{\clearpage
  \@mainmatterfalse
  \pagenumbering{Roman}} % 正文前罗马字体编号
\makeatother


%%%%%%%%%% References %%%%%%%%%%%%%%%%%
\renewcommand{\bibname}{参考文献}
% 重定义参考文献样式,来自thu
\makeatletter
\renewenvironment{thebibliography}[1]{%
   \chapter*{\bibname}%
   \wuhao
   \list{\@biblabel{\@arabic\c@enumiv}}%
        {\renewcommand{\makelabel}[1]{##1\hfill}
         \setlength{\baselineskip}{21pt}
         \settowidth\labelwidth{0.5cm}
         \setlength{\labelsep}{0pt}
         \setlength{\itemindent}{0pt}
         \setlength{\leftmargin}{\labelwidth+\labelsep}
         \addtolength{\itemsep}{-0.7em}
         \usecounter{enumiv}%
         \let\p@enumiv\@empty
         \renewcommand\theenumiv{\@arabic\c@enumiv}}%
    \sloppy\frenchspacing
    \clubpenalty4000%
    \@clubpenalty \clubpenalty
    \widowpenalty4000%
    \interlinepenalty4000%
    \sfcode`\.\@m}
   {\def\@noitemerr
     {\@latex@warning{Empty `thebibliography' environment}}%
    \endlist\frenchspacing}
\makeatother

\addtolength{\bibsep}{5pt} % 增加参考文献间的垂直间距
\setlength{\bibhang}{2em} %每个条目自第二行起缩进的距离

% 参考文献引用作为上标出现
\newcommand{\mycite}[1]{\scalebox{1.3}[1.3]{\raisebox{-0.65ex}{\cite{#1}}}}
%\newcommand{\citenormal}[1]{\cite{#1}}
%\makeatletter
%   \def\@cite#1#2{\textsuperscript{[{#1\if@tempswa , #2\fi}]}}
%\makeatother

%% 引用格式
\bibpunct{[}{]}{,}{s}{}{,}

%%%%%%%%%% Cover %%%%%%%%%%%%%%%%%
% 封面、摘要、版权、致谢格式定义
\makeatletter

\def\dtitle#1{\def\@dtitle{#1}}\def\@dtitle{}
\def\ctitle#1{\def\@ctitle{#1}}\def\@ctitle{}
\def\etitle#1{\def\@etitle{#1}}\def\@etitle{}
\def\caffil#1{\def\@caffil{#1}}\def\@caffil{}
\def\cmacrosubject#1{\def\@cmacrosubject{#1}}\def\@cmacrosubject{}
\def\cmacrosubjecttitle#1{\def\@cmacrosubjecttitle{#1}}\def\@cmacrosubjecttitle{}
\def\csubject#1{\def\@csubject{#1}}\def\@csubject{}
\def\csubjecttitle#1{\def\@csubjecttitle{#1}}\def\@csubjecttitle{}
\def\cmajor#1{\def\@cmajor{#1}}\def\@cmajor{}
\def\cauthor#1{\def\@cauthor{#1}}\def\@cauthor{}
\def\cauthortitle#1{\def\@cauthortitle{#1}}\def\@cauthortitle{}
\def\csupervisor#1{\def\@csupervisor{#1}}\def\@csupervisor{}
\def\ocsupervisor#1{\def\@ocsupervisor{#1}}\def\@ocsupervisor{}
\def\csupervisortitle#1{\def\@csupervisortitle{#1}}\def\@csupervisortitle{}
\def\cdate#1{\def\@cdate{#1}}\def\@cdate{}
\def\untitle#1{\def\@untitle{#1}}\def\@untitle{}
\def\declaretitle#1{\def\@declaretitle{#1}}\def\@declaretitle{}
\def\declarecontent#1{\def\@declarecontent{#1}}\def\@declarecontent{}
\def\authorizationtitle#1{\def\@authorizationtitle{#1}}\def\@authorizationtitle{}
\def\authorizationcontent#1{\def\@authorizationcontent{#1}}\def\@authorizationconent{}
\def\authorizationadd#1{\def\@authorizationadd{#1}}\def\@authorizationadd{}
\def\authorsigncap#1{\def\@authorsigncap{#1}}\def\@authorsigncap{}
\def\supervisorsigncap#1{\def\@supervisorsigncap{#1}}\def\@supervisorsigncap{}
\def\signdatecap#1{\def\@signdatecap{#1}}\def\@signdatecap{}
\long\def\cabstract#1{\long\def\@cabstract{#1}}\long\def\@cabstract{}
\long\def\eabstract#1{\long\def\@eabstract{#1}}\long\def\@eabstract{}
\def\ckeywords#1{\def\@ckeywords{#1}}\def\@ckeywords{}
\def\ekeywords#1{\def\@ekeywords{#1}}\def\@ekeywords{}
\def\cheading#1{\def\@cheading{#1}}\def\@cheading{}
\def\cnumber#1{\def\@cnumber{#1}}\def\@cnumber{}
\def\csecret#1{\def\@csecret{#1}}\def\@csecret{}
\def\chnunumer#1{\def\@chnunumer{#1}}\def\@chnunumer{}
\def\cclassnumber#1{\def\@cclassnumber{#1}}\def\@cclassnumber{}
\def\chnuname#1{\def\@chnuname{#1}}\def\@chnuname{}
\def\cchair#1{\def\@cchair{#1}}\def\@cchair{}
\def\ddate#1{\def\@ddate{#1}}\def\@ddate{}
%英文内封
\def\ename#1{\def\@ename{#1}}\def\@ename{}
\def\cbe#1{\def\@cbe{#1}}\def\@cbe{}
%\def\cms#1{\def\@cms{#1}}\def\@cms{}
\def\cdegree#1{\def\@cdegree{#1}}\def\@cdegree{}
\def\cclass#1{\def\@cclass{#1}}\def\@cclass{}
% \def\emajor#1{\def\@emajor{#1}}\def\@emajor{}
\def\ehnu#1{\def\@ehnu{#1}}\def\@ehnu{}
\def\esupervisor#1{\def\@esupervisor{#1}}\def\@esupervisor{}
\def\edate#1{\def\@edate{#1}}\def\@edate{}
\def\elevel#1{\def\@elevel{#1}}\def\@elevel{}


%%%%%%%%%%%%%%%%%%%   封面   %%%%%%%%%%%%%%%%%%%%%%%

\newlength{\@title@width}
\def\@put@covertitle#1{\makebox[\@title@width][s]{#1}}
% 定义封面
\def\makecover{
%\cleardoublepage%
  \phantomsection
  \pdfbookmark[-1]{\@ctitle}{ctitle}

  \begin{titlepage}
  \begin{center}

  \setlength{\@title@width}{3.5cm}
  {
    % \setlength{\tabcolsep}{6pt}  % 列间距
    \renewcommand{\arraystretch}{1.5}  % 行间距
    \begin{tabular}{lcclc}
    \xiaosi\hei{学校代号}&  \underline{\makebox[\@title@width][c]{\xiaosi\hei\@chnunumer}}&\qquad \qquad \qquad \qquad \qquad & \xiaosi\hei{学\qquad 号}&  \underline{\makebox[\@title@width][c]{\xiaosi\hei\@cnumber}}\\ 
    \xiaosi\hei{分~~类~~号}&  \underline{\makebox[\@title@width][c]{\xiaosi\hei\@cclassnumber}}&\qquad \qquad \qquad \qquad \qquad & \xiaosi\hei{密\qquad 级}&  \underline{\makebox[\@title@width][c]{\xiaosi\hei\@csecret}} \\
     & & & & \\ % 用于空行
     & & & & \\
    \end{tabular}
  }

  \begin{figure}[h]
  \centering
  \includegraphics[width=0.5\textwidth]{figures/cslglogo.png}
  \end{figure}
  \vspace*{1cm}
  {\hei\erhao \@cheading}

  \vspace*{1cm}

  \begin{center}
  \begin{spacing}{1.5}
  %\hei\yihao \@dtitle%原标题断字会影响页眉
  \hei\erhao \@dtitle%标题可以使用断字
  \end{spacing}
  \end{center}
  %\makebox[宽度][位置]{文本}中可指定盒子宽度,文本在盒子中的位置(l:左端;r:右端;s:两端,默认是居中)
  \vspace{\baselineskip}
  \setlength{\@title@width}{6.5cm}
  {
  \begin{spacing}{2.1}
	\makebox[3.3cm][s]{\xiaosi\hei{学位申请人姓名}}\xiaosi\song\underline{\makebox[\@title@width][c]{\@cauthor}} \\
	\makebox[3.3cm][s]{\xiaosi\hei{所~~~~在~~~~学~~~~校}} \xiaosi\song\underline{\makebox[\@title@width][c]{\@caffil}} \\
	\makebox[3.3cm][s]{\xiaosi\hei{专~业~学~位~类~别}} \xiaosi\song\underline{\makebox[\@title@width][c]{\@csubject}} \\
	\makebox[3.3cm][s]{\xiaosi\hei{专~业~学~位~领~域}} \xiaosi\song\underline{\makebox[\@title@width][c]{\@cmajor}} \\
	% \makebox[3.3cm][s]{\xiaosi\hei{导师姓名及职称}} \xiaosi\song\underline{\makebox[\@title@width][c]{\@csupervisor}} \\
	\makebox[3.3cm][s]{\scalebox{0.84}[1]{\xiaosi\hei{校内导师姓名及职称}}} \xiaosi\song\underline{\makebox[\@title@width][c]{\@csupervisor}} \\
	\makebox[3.3cm][s]{\scalebox{0.84}[1]{\xiaosi\hei{校外导师姓名及职称}}} \xiaosi\song\underline{\makebox[\@title@width][c]{\@ocsupervisor}} \\
	\makebox[3.3cm][s]{\xiaosi\hei{论~文~提~交~日~期}} \xiaosi\song\underline{\makebox[\@title@width][c]{\@cdate}} \\
  \end{spacing}
  }
  \end{center}

%%%%%%%%%%%%%%%%%%%   中文内封   %%%%%%%%%%%%%%%%%%%%%%%

\clearpage
\thispagestyle{empty} %去掉页眉页脚

\noindent
% \makebox[2.59cm][s]{}
{
\renewcommand{\arraystretch}{1.5}  % 行间距
\begin{tabular}{ll}
\makebox[2cm][s]{\xiaosi\hei 学校代号:}\xiaosi\hei~~\@chnunumer \\
\makebox[2cm][s]{\xiaosi\hei 学\qquad~号:}\xiaosi\hei~~\@cnumber\\
\makebox[2cm][s]{\xiaosi\hei 密\qquad~级:}\xiaosi\hei~~\@csecret\\
\end{tabular}
}

\begin{center}

  \vspace{2\baselineskip}
  {\xiaoer\song\bf \@chnuname \@cheading}  % bf加粗
  \vspace*{1cm}

  \begin{center}
  \begin{spacing}{1.5}
  \hei\erhao \@dtitle
  \end{spacing}
  \end{center}
\end{center}
\vspace{6\baselineskip}

%%%%%%%

\setlength{\@title@width}{5.5cm}
{
  \begin{spacing}{2}
  \xiaosi
  \noindent
  % \makebox[2.59cm][s]{}
  \begin{center}
  {
    \begin{tabular}{lc}
      \makebox[3.3cm][s]{\xiaosi\hei学位申请人姓名}\underline{\song\makebox[\@title@width][c]{\@cauthor}} \\
      \makebox[3.3cm][s]{\xiaosi\hei所~~~~在~~~~学~~~~院}\underline{\song\makebox[\@title@width][c]{\@caffil}} \\
      \makebox[3.3cm][s]{\xiaosi\hei导师姓名及职称}\underline{\song\makebox[\@title@width][c]{\@csupervisor}} \\
      \makebox[3.3cm][s]{\xiaosi\hei专~业~学~位~领~域}\underline{\song\makebox[\@title@width][c]{\@csubject}} \\
      \makebox[3.3cm][s]{\xiaosi\hei论~文~提~交~日~期}\underline{\song\makebox[\@title@width][c]{\@cdate}} \\
      \makebox[3.3cm][s]{\xiaosi\hei论~文~答~辩~日~期}\underline{\song\makebox[\@title@width][c]{\@ddate}}\\
      \makebox[3.3cm][s]{\xiaosi\hei答辩委员会主席}\underline{\song\makebox[\@title@width][c]{\@cchair}} \\
    \end{tabular}
  }
  \end{center}
  \end{spacing}
}

%%%%%%%%%%%%%%%%%%%   英文内封   %%%%%%%%%%%%%%%%%%%%%%%

\clearpage
\thispagestyle{empty} %去掉页眉页脚

\begin{center}
  \qquad\\
  \qquad\\
  \begin{spacing}{1.5}
  \xiaosan \@etitle
  \end{spacing}

  \qquad\\
  \qquad\\
  \begin{spacing}{2}
  \xiaosi
  by\\
  \@ename \\

  \qquad\\
  \@cbe\\
  %\@cms\\
  A~\@cdegree~submitted in partial satisfaction of the\\

  \qquad\\
  Requirements for the degree of\\
  \@cclass\\

  \vspace{4\baselineskip}
  in\\
  %  \@emajor\\
  %  in the\\
  %  Graduate School\\
  %  of\\
  \@ehnu\\
  Supervisor\\
  \@elevel~~ \@esupervisor\\
  \@edate
  \end{spacing}
\end{center}

\end{titlepage}

%  另起一页: 独创性声明和学位论文版权使用授权书
% 如果需要上传稿包含版权页,取消这部分内容注释
%\addcontentsline{toc}{chapter}{学位论文原创性声明和学位论文版权使用授权书}
%\setcounter{page}{1}
%\includepdf{Copyright.pdf}
% 如果需要打印稿,即需要打印后手写,取消该部分内容注释
% \pagestyle{fancy}
% \fancyhf{}
% \fancyfoot[C]{\song\xiaowu ~\thepage~}
% \renewcommand{\headrulewidth}{0pt}

\clearpage
\thispagestyle{empty} %去掉页眉页脚

% \addcontentsline{toc}{chapter}{学位论文原创性声明和学位论文版权使用授权书}
{
% \setcounter{page}{1}
\qquad\\
\begin{center}{\hei\xiaosi\bf \@untitle}\end{center}\par
\begin{center}{\hei\xiaosi\bf \@declaretitle}\end{center}\par
\song\defaultfont{\@declarecontent}\par
\vspace*{1cm}
{\song\xiaosi
\@authorsigncap \makebox[3.5cm][s]{}
\@signdatecap \makebox[1.5cm][s]{} 年 \makebox[1cm][s]{} 月 \makebox[1cm][s]{} 日
}
\vspace{2\baselineskip}
\begin{center}{\hei\xiaosi\bf \@authorizationtitle}\end{center}\par
{
\song\defaultfont{\@authorizationcontent}\\
% \song\xiaoxiao{ \quad }\\
\begin{table}[h]
\vskip -0.40in
\renewcommand{\arraystretch}{1.95}
\begin{tabular}{ll}
\song\defaultfont\@authorizationadd\par&\\
&1、保密\song\xiaoer{$\Box$}\song\xiaosi ,在\underline{\qquad}年解密后适用于本授权书\\
&2、不保密\song\xiaoer{$\Box$}。\\
&(请在以上相应方框内打“$\surd$”) \\
\end{tabular}
\vskip -0.40in
\end{table}
}
\vspace{1.5\baselineskip}

{
\song\xiaosi
\@authorsigncap \makebox[3.5cm][s]{}  \@signdatecap \makebox[1.5cm][s]{} 年 \makebox[1cm][s]{} 月 \makebox[1cm][s]{} 日 \\
\vspace{0.6\baselineskip}\\
\indent
\@supervisorsigncap \makebox[3.5cm][s]{}  \@signdatecap \makebox[1.5cm][s]{} 年 \makebox[1cm][s]{} 月 \makebox[1cm][s]{} 日
}
}


%%%%%%%%%%%%%%%%%%%   Abstract and Keywords  %%%%%%%%%%%%%%%%%%%%%%%
\clearpage
\pagestyle{fancy}
\fancyhf{}
\fancyfoot[C]{\song\xiaowu ~\thepage~}
\renewcommand{\headrulewidth}{0pt}

% \pagestyle{fancy}
%   \fancyhf{}
% \fancyhead[CO]{\song\wuhao \leftmark}
% \fancyhead[CE]{\song\wuhao \@ctitle}
% \fancyfoot[C]{\song\wuhao ~\thepage~}
% \def\headrule{{\if@fancyplain\let\headrulewidth\plainheadrulewidth\fi%
% \hrule\@height 1.0pt \@width\headwidth\vskip1pt %上面线为1pt粗
% \hrule\@height 0.5pt\@width\headwidth  %下面0.5pt粗
% \vskip-2\headrulewidth\vskip-1pt}      %两条线的距离1pt
% \vspace{7mm}
% }     %双线与下面正文之间的垂直间距


% \pagestyle{fancy}
% \fancyhf{} % 清空页眉页脚
% \fancyhead[CO]{\song\wuhao ~\leftmark~}
% \fancyhead[CE]{\song\wuhao \@cheading}
% \fancyfoot[C]{\song\wuhao ~\thepage~}
% \renewcommand{\headrulewidth}{1.5pt}

% \fancypagestyle{plain}{% 设置开章页页眉页脚风格
% \fancyhf{}%
% \fancyhead[CO]{\song\wuhao ~\leftmark~}
% \fancyhead[CE]{\song\wuhao \@cheading}
% \fancyfoot[C]{\song\wuhao ~\thepage~}
% }

\pagestyle{fancy}
\fancyhf{} 
\fancyfoot[C]{\song\xiaowu ~\thepage~}
\renewcommand{\headrulewidth}{0pt}

\setcounter{page}{1}
\addcontentsline{toc}{chapter}{摘\quad 要}
\chapter*{\centering\xiaoer\ 摘\qquad 要}
\song\defaultfont
\@cabstract
\vspace{\baselineskip}

%\hangafter=1\hangindent=52.3pt\noindent   %如果取消该行注释,关键词换行时将会自动缩进
\noindent
{\hei\xiaosi 关键词: \@ckeywords}

%%%%%%%%%%%%%%%%%%%   English Abstract   %%%%%%%%%%%%%%%%%%%%%%%%%%%%%%
\clearpage
\pagestyle{fancy}
\fancyhf{}
\fancyfoot[C]{\song\xiaowu ~\thepage~}
\renewcommand{\headrulewidth}{0pt}

\addcontentsline{toc}{chapter}{Abstract}
\chapter*{\centering\xiaoer \bf{Abstract}}
%\vspace{\baselineskip}
\@eabstract
\vspace{\baselineskip}

%\hangafter=1\hangindent=60pt\noindent  %如果取消该行注释,KEY WORDS换行时将会自动缩进
\noindent
{\xiaosi\textbf{Key Words:} \@ekeywords}
}

\clearpage
\makeatother                       % 完成对论文各个部分格式的设置
\frontmatter                               % 以下是论文导言部分,包括论文的封面,中英文摘要和中文目录
% !Mode:: "TeX:UTF-8"

% """
% Edited on 03/10/2023
% anoi.
% @author: Kang Xiatao (kangxiatao@gmail.com)
% """

\chnunumer{10536}
\chnuname{长沙理工大学}
\cclassnumber{TP391}
\cnumber{20208051427}
% \cnumber{*}
\csecret{公开}
\cmajor{工程硕士}  % 学位类别
\cdegreethesis{专业硕士学位论文}      % 自己的学位论文级别
\cheading{硕士学位论文}      % 设置正文的页眉
\dtitle{神经网络剪枝与稀疏模型泛化研究}%封面用论文标题,自己可手动断行\\
\ctitle{神经网络剪枝与稀疏模型泛化研究}  %页眉标题无需断行
\etitle{Research on Neural Network Pruning and Sparse Model Generalization}
\caffil{计算机与通信工程学院} %学院名称
\csubjecttitle{学科专业}
\csubject{电子信息}   %学位领域
\cauthortitle{研究生}     % 学位
\cauthor{康夏涛}   %学生姓名
% \cauthor{*}   %学生姓名
\ename{KANG~Xiatao}
% \ename{*}
\cbe{B.E.~(Hubei Polytechnic University)~2020}
% \cms{M.S.~(University)~2020}
\cdegree{thesis}
\cclass{Master of engineering}
% \emajor{Computer Science and Technology}
\ehnu{Changsha University of Science \& Technology}
\esupervisor{Li Ping}
% \esupervisor{*}
\csupervisortitle{指导教师}
\csupervisor{李平~~教授} %导师姓名
% \csupervisor{*} %导师姓名
\elevel{Professor} %导师职称
\ocsupervisor{曾彬~~高级工程师} %校外导师
\cchair{肖晓丽}
\ddate{2023年5月} %论文答辩日期
\edate{April,~2023}

\untitle{长沙理工大学}
\declaretitle{学位论文原创性声明}
\declarecontent{
    本人郑重声明:所呈交的论文是本人在导师的指导下独立进行研究所取得的研究成果。除了文中特别加以标注引用的内容外,本论文不包含任何其他个人或集体已经发表或撰写的成果作品。对本文的研究做出重要贡献的个人和集体,均已在文中以明确方式标明。本人完全意识到本声明的法律后果由本人承担。
}
\authorizationtitle{学位论文版权使用授权书}
\authorizationcontent{
    本学位论文作者完全了解学校有关保留、使用学位论文的规定,同意学校保留并向国家有关部门或机构送交论文的复印件和电子版,允许论文被查阅和借阅。本人授权长沙理工大学可以将本学位论文的全部或部分内容编入有关数据库进行检索\scalebox{0.9}[1]{,}可以采用影印、缩印或扫描等复制手段保存和汇编本学位论文。同时授权中国科学技术信息研究所将本论文收录到《中国学位论文全文数据库》,并通过网络向社会公众提供信息服务。
}
\authorizationadd{本学位论文属于}
\authorsigncap{作者签名:}
\supervisorsigncap{导师签名:}
\signdatecap{日期:}


%\cdate{\CJKdigits{\the\year} 年\CJKnumber{\the\month} 月 \CJKnumber{\the\day} 日}
% 如需改成二零一二年四月二十五日的格式,可以直接输入,即如下所示
\cdate{2023年4月} %论文提交日期
% \cdate{\the\year 年\the\month 月 \the\day 日} % 此日期显示格式为阿拉伯数字 如2012年4月25日
\cabstract{
    大规模神经网络模型涌现出了让人类叹为观止的能力,但随之而来的是不可解释性和千亿级参数量,这使得相关研究比以往任何时候都更具挑战性。最近的一些研究从神经网络剪枝的角度来探讨网络的运行机理,而不是仅仅为了模型压缩。主要的工作集中在训练前剪枝和稀疏模型深层结构研究,且训练前修剪模型后在训练阶段也能起到加速作用,有很强的应用价值。本文以提升稀疏模型泛化为目标,对训练前剪枝展开研究,探讨了在每轮训练中由一批样本训练引起待训练样本的损失隐式减少的新视角,提出了在稀疏网络过程中权重表现力转移的概念。

    对于隐式损失下降,本文给出了其称为梯度耦合流的一阶近似,并探索了耦合流与稀疏模型泛化之间的联系,隐式损失理想地以细粒度的方式描述了精度波动的原因。由隐式损失下降的特性,本文提出敏感于梯度耦合流的权重度量准则,在初始化时捕获那些对性能提升最敏感的权重。
    
    通过权重表现力转移的性质,本文对度量指标做动态分析,发现若要维持网络原有的性能,剩余权重将承担来自删减权重的表现力。为了实现最优的表现力调度,本文提出称为淘金的训练前修剪方案,通过多指标多流程的步骤引导表现力转移,并设计强化学习智能体实现淘金策略的自动化。
    
    另外,本文的梯度耦合流敏感度量和强化学习淘金方案都在图像分类任务中进行了实证研究。梯度耦合流敏感在训练前的单次剪枝和迭代剪枝中都有优异的表现,而且耦合流的细分扩展更好的证明了耦合流的有效。强化学习淘金取得了非常好的稀疏效果,在任意压缩率下呈现了不错的性能,且可扩展并适用于大规模模型和数据集。实证表明,梯度耦合流和权重作用力为研究网络运行机理提供了有效的方法,并对提高过参数化网络可解释性和稳定性做出了贡献。
}

\ckeywords{神经网络剪枝;~~稀疏模型泛化;~~训练前剪枝;~~强化学习剪枝;~~梯度耦合流;~~权重表现力}

\eabstract{
    The emergence of large-scale neural network models has produced astonishing capabilities, but it also brings with it uninterpretability and hundreds of billions of parameters, which makes related research more challenging than ever. Recent research has explored the operating mechanism of neural networks from the perspective of network pruning, rather than simply compressing models. The main focus is on pruning before training and structural research on deep sparse models, and pruning models before training can also accelerate the training process, with strong practical value. This dissertation aims to improve the generalization of sparse models, conducts research on pruning before training, explores a new perspective on loss implicit decrease of the data to be trained caused by one-batch training during each round, and proposes the concept of weight expressive force transfer in sparse network processes.

    For the implicit loss reduction, this dissertation provides its first-order approximation called gradient coupled flow, and explores the relationship between coupled flow and the generalization of sparse models. The implicit loss ideally describes the cause of accuracy fluctuations in a fine-grained manner. Based on the characteristics of implicit loss reduction, this dissertation proposes a weight measure criterion sensitive to gradient coupled flow, which captures the weights that are most sensitive to performance improvement at initialization.

    Based on the perspective of weight expressive force transfer, this dissertation analyzes the measurement indicators and finds that the remaining weights will bear the expressive force from the deleted weights in order to maintain the network original performance. In order to achieve optimal expressive force scheduling, this dissertation proposes a pruning before training scheme called panning, which guides the transfer of expressive force through a multi-indicator and multi-process step, and designs an intelligent agent based on reinforcement learning to automate this process.

    In addition, the gradient coupled flow sensitivity measure and the reinforcement learning panning scheme proposed in this dissertation were empirically studied in image classification tasks. The gradient coupled flow sensitivity measure performed well in both single-shot and iterative pruning before training, and the finer subdivision of the coupled flow better demonstrated its effectiveness. The reinforcement learning panning achieved very good sparse effects, showing excellent performance at any compression rate and being applicable to large-scale models and datasets. The empirical results show that the gradient coupled flow and weight expressive force transfer provide effective methods for studying the operation mechanism of neural networks, and contribute to improving the interpretability and stability of over-parameterized networks.
}

\ekeywords{neural network pruning;~~sparse model generalization;~~pruning before training;~~reinforcement learning pruning;~~gradient coupled flow;~~weight expressive force}

\makecover

\clearpage
                      % 封面


%%%%%%%%%%   目录   %%%%%%%%%%
\clearpage{\thispagestyle{empty}\cleardoublepage}  % 空白页,控制下一页为奇数页
\sk20font
\addcontentsline{toc}{chapter}{目~~~~录}
\tableofcontents                           % 中文目录
\clearpage{\thispagestyle{empty}\cleardoublepage}  % 空白页,控制下一页为奇数页
\newcommand{\loflabel}{图~}
\renewcommand{\numberline}[1]{\song\xiaosi\loflabel~#1\hspace*{\baselineskip}}
\addcontentsline{toc}{chapter}{插图索引}
\listoffigures
\clearpage{\thispagestyle{empty}\cleardoublepage}  % 空白页,控制下一页为奇数页
\newcommand{\lotlabel}{表~}
\renewcommand{\numberline}[1]{\song\xiaosi\lotlabel~#1\hspace*{\baselineskip}}
\addcontentsline{toc}{chapter}{附表索引}
\listoftables
\clearpage{\pagestyle{empty}\cleardoublepage}


%%%%%%%%%% 正文部分内容  %%%%%%%%%%
\mainmatter\defaultfont\sloppy\raggedbottom
\makeatletter
\renewcommand{\ALC@linenosize}{\xiaosi}
\clearpage

% 这里有bug,我吐了,\leftmark中莫名有编号,只能在对应章节里手动修改页眉
\pagestyle{fancy} % 设置页眉页脚风格
\fancyhf{} % 清空页眉页脚
\fancyhead[CO]{\song\wuhao ~\leftmark~} % !!!
\fancyhead[CE]{\song\wuhao \@cheading}
\fancyfoot[C]{\song\wuhao ~\thepage~}
\renewcommand{\headrulewidth}{1.5pt}
\fancypagestyle{plain}{% 设置开章页页眉页脚风格
\fancyhf{}%
\fancyhead[CO]{\song\wuhao ~\leftmark~}
\fancyhead[CE]{\song\wuhao \@cheading}
\fancyfoot[C]{\song\wuhao ~\thepage~}}

\makeatother

% \renewcommand\arraystretch{1.5}
\setlength{\intextsep}{2pt}
\setlength{\abovecaptionskip}{2pt}
\setlength{\belowcaptionskip}{2pt}
\include{body/intros}
\include{body/figures}
\include{body/tables}
\include{body/equations}
\include{body/others}
\include{body/conclusion}
%%%%%%%%%% 正文部分内容  %%%%%%%%%%


%%%%%%%%%%  参考文献  %%%%%%%%%%
\clearpage{\thispagestyle{empty}\cleardoublepage}  % 空白页,控制下一页为奇数页
\fancyhead[CO]{\song\wuhao ~参考文献~}  % 手动修改页眉
\fancypagestyle{plain}{\fancyhead[CO]{\song\wuhao ~参考文献~}}

\defaultfont
\bibliographystyle{gbt7714-numerical}
\phantomsection
\addcontentsline{toc}{chapter}{参考文献}          % 参考文献加入到中文目录
%\nocite{*}                                        % 若将此命令屏蔽掉,则未引用的文献不会出现在文后的参考文献中。


%%%%%%%%%%  致谢和附录  %%%%%%%%%%
\clearpage{\thispagestyle{empty}\cleardoublepage}  % 空白页,控制下一页为奇数页
\bibliography{references/reference}
\clearpage{\thispagestyle{empty}\cleardoublepage}  % 空白页,控制下一页为奇数页
% !Mode:: "TeX:UTF-8"

% 手动修改页眉
\fancyhead[CO]{\song\wuhao ~致谢~}
\fancypagestyle{plain}{\fancyhead[CO]{\song\wuhao ~致谢~}}

\addcontentsline{toc}{chapter}{致\quad 谢}%添加到目录中
\chapter*{致\quad 谢}

所谓的成长,不过是燃烧自己的无知,然后逐渐麻木。

恍惚间已是三年,伴随着疫情和动荡,在毕业之际,真心的感谢祖国,感谢父母,感谢抗疫前线的英雄们,感谢所有人的牺牲和付出换来形式的向好,让我们生活步入正轨,安稳的完成学业。

古之学者必有师,感谢我亦师亦友的导师李平教授。在路灯下讨论方案、在食堂外分析实验、在教室里推导公式、在民宿中撰写论文,这些画面一帧一帧的浮现出来,成了我研究生期间最重要的记忆。在李老师将下,我跟进了最前沿的技术、丰富了理论知识和思维、参加了学术会议,为我的人生写了一段关于科研魅力的故事。

风华正茂,感谢这些年挚友的关怀和帮助。特别感谢~OYX~为我提供丰富的高校资源,感谢~XZH~为我讲解算法和部署实践,感谢~KZC~在实验上提供的灵感和他仅有的显卡,感谢~AF~分享的文献和调参经验。

修辞立其诚,实验立其真。作为人工智能领域的学者,非常感谢~Kaggle~、~Google Colab~、阿里云天池、华为云AI提供的云计算资源。感谢~DeepMind~、~OpenAI~和~FAIR~提供的开源代码和技术文档。更感谢这几年所有前辈的努力,我们拥有一个很友好的深度学习开发环境。

为学生二十余载,感谢我终迎来了最后一次毕业。我很好学,但终究是做不来学者,这条路也算是走到头了。在研究生期间,在这个历史的冬天,在这个卷不赢躺不平的时代,我的哲思过于极端,以至于都懒得去建设短暂的同学情谊。很长一段时间我都在找一个答案,一个现代主义没有给出的答案。逐渐我也有了答案,本以为的麻木,殊不知为无用之用。

一树百获的三年,虽行久矣,而方始发,任重道远。或某日同风起,直上九万里。

\rightline{康夏涛}
\rightline{二零二三年四月}
\rightline{于长沙理工}









               % 致谢
\clearpage{\thispagestyle{empty}\cleardoublepage}  % 空白页,控制下一页为奇数页
\include{appendix/publications}                   % 发表论文和参加科研情况说明
\clearpage{\thispagestyle{empty}\cleardoublepage}  % 空白页,控制下一页为奇数页


\clearpage
\end{CJK*}                                        % 结束中文字体使用
\end{document}                                    % 结束全文
